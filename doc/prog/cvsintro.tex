\section{Managing NWChem}
\label{sec:cvs-intro}

{\it Concurrent Version System} (CVS) is used for configuration management of
NWChem at PNNL.  Off-site users are not {\em required} to used this system when 
doing development work on the code, but it would probably make any collaborative
work with EMSL/PNNL developers go much more smoothly.  As a matter of simple
prudence, is advisable to use some sort
of configuration management system for any installation of NWChem, even if
the users do not expect to be doing significant development work.  The code
is far too complex to ever be released on it's own recognizance, and users
will want to retain the ability to correct errors or make modifications to the
code in a controlled and traceable manner.  

CVS was chosen as the configuration management system for NWChem mainly
because is designed to allow many different developers to work independently
on a large code, while greatly mitigating the agony of merging independently
developed sets of code changes.  For developers working on unrelated modules
of the code, the effects of changes made elsewhere in the code can in some cases
be completely ignored.  The identification of overlapping changes is greatly
facilitated, allowing efficient and speedy resolution of conflicts.

This section provides a brief introduction and overview of the CVS system.
Developers needing more detailed information on specific CVS commands and
capabilities are refered to the on-line documentation included in the CVS 
code package (i.e., the \texttt{man} pages; consult your system administrator
if your system does not have them installed.)

\subsection{Introduction to CVS}

% {\bf Author:} David Bernholdt (modified by JM Cuta, 8/12/98)

CVS is a configuration
control package designed to facilitate multiple developers working on
the same software package.  It is implemented as a layer on top of RCS
and provides a number of useful features which RCS alone does not.
The two most important of these features are 
\begin{itemize}
\item The CVS check-in/check-out mechanism does not require exclusive
locks on sources during the development process, and provides for
merging of orthogonal changes to the same source file.  (Overlapping
changes are identified during the merger process, and must be resolved
by human intervention.)
\item  Most CVS commands work recursively
on the entire contents of a directory tree, unless specific command line
switches are set to limit operation to the local directory.
\end{itemize}
The following subsections provides a brief description of how NWChem is managed in
CVS.  It also includes a very concise outline of how CVS works, and a
summary of the most useful CVS commands.

\subsection{The CVS Model}

CVS divorces the directory tree in which development takes place from
the directory tree in which the master copy of the sources are kept.
The latter directory tree is referred to as the {\em repository}, and
it has exactly the same structure as the working directory tree.  Where
the working tree would have source files, the repository has the RCS
files for the sources (e.g., {\tt source.f,v}).

Users working on a program check it out of the repository
into their own directories.  The individual working copies are by
default created giving the user read and write permission on all of
the files and can be used directly.  When a developer has completed
and tested a set of changes, the revised source file(s) can be checked into
the repository.  The other developers are unaffected by the change to
the repository until they update their local copy of the source or
check out a new copy.  Anyone checking out a new working copy will
always get the latest version present in the repository.

Users can poll the repository for changes at any time, and update {\em
their own working copies} with the changes that have been entered in the
interval between their last checkout or
update and the current version.  The repository is entirely unaffected
by the update command.  The user's private working copy is the only
thing that is changed.  If any changes were merged into the repository
between the user's last check-out or update and the current one,
dealing with any inconsistencies or overlaps with changes in the
local working copies is the user's problem.

When a user checks a revised source file back into the repository, 
CVS automatically checks for all differences between the copy being checked
in and the current version of the file in the repository.  If changes in
the new file being checked in overlap or conflict with changes that have
been merged into the repository since the last check-out or update of the
copy being merged, CVS will not automatically merge the new copy into the repository.
If the changes do not overlap or conflict, however, CVS will merge the new
source over the existing source in the repository.

In most cases, changes made
independently by different developers will not conflict and CVS can
handle the merger automatically.  When they do conflict, 
the developer must fix the problem(s) and ensure that the new changes
mesh properly with changes others have put into the repository.
CVS allows users to work independently on the same source files without
unduly interfering with each other, but it is still necessary for 
developers working on functionally related changes to communicate with
each other, even if their source code changes do not conflict. 

\subsection{The CVS Program}

CVS is implemented as a single program invoked by its program name {\tt cvs}.
A number of options can be specified on the command line following the
program name.  The command line can also include subcommands, which
come after any options that may be specified.  The syntax of the command
line is as follows;

\begin{verbatim}
cvs [cvs_options] subcommand [subcommand_options] [arguments]
\end{verbatim}

The man pages list the
applicable options for the cvs command itself and for each subcommand.

CVS must be told of the location of the repository.  This can be done
with the cvs\_option {\tt -d} (e.g., {\tt -d /msrc/proj/mss}) or by setting
the environment variable {\tt CVSROOT}.  Although the CVS man pages
implicitly assume that a single repository will be used 
for all projects under CVS control,
this is not strictly necessary.  Different repositories can be defined
by the simple expedient of changing the definition of {\tt CVSROOT}.

CVS is designed to deal with source files organized into {\em modules}.
A module is basically a
collection of source files that form some sort of sensible unit and probably should
be worked on as a group.  The module can
simply be the name of a directory within the repository (e.g. {\tt
nwchem} or {\tt nwchem/src}), or it is defined as a collection of selected
bits and pieces of the directories within the repository.  For
example, it might eventually be desirable allow users to check out
NWChem without getting certain parts of the package, such as
Argos sources or the distributed data package.  Specific modules
could be defined to give these results.

The procedure for checking out a working copy of the code stored in
CVS repository is very simple.  From the directory where the working
copy is to be checked out to, a given module can be checked out using the following
command;

\begin{verbatim}
cvs co module_name
\end{verbatim}


To check out NWChem, the command is simply,


\begin{verbatim}
cvs co nwchem
\end{verbatim}


The working version of a module in a local directory can be compared
with the source in the repository using the command

\begin{verbatim}
cvs diff
\end{verbatim}

This command accepts the same arguments as {\tt rcsdiff}, and will compare 
particular files itemized on the command line or the entire directory tree recursively.
(The command {\tt cvs log} is the equivalent of the RCS rlog command and operates
similarly to {\tt cvs diff}.)

Changes made to the repository after a particular working copy has been checked out
can be merged into the files on the working directory using the command

\begin{verbatim}
cvs update
\end{verbatim}

This command is recursive throughout the checked-out directory tree.  It flags
modified files in the working directory  with an "M".  Files that have
changed in the repository since the last update are marked with a "U".  
New files in the working directory that do not occur in the repository
are marked with a ``?''.  There are a
number of other codes for other circumstances, which are detailed in
the man pages. A particularly useful command is the option to check on what
has changed since the last update of the working directory, but without
merging any of the changes from the repository.  This can be done using the command

\begin{verbatim}
cvs -n update
\end{verbatim}

To remove a file from a repository controlled by CVS
it first must be removed from the directory with the Unix {\tt rm} command.  The
command {\tt cvs rm} is then used to notify CVS\@.  When this (nonexistent) file
is checked in at the next update,
it will be moved to a special place in the repository where
it can be recovered if old versions which require it are checked out,
but where it will not appear in future working copies.

To add a new file to a repository controlled by CVS, 
the command is {\tt cvs add}.  Like {\tt cvs
rm}, the actual addition takes place at the next check-in.  As with
the first RCS check-in of a file, {\tt cvs add} will prompt for a
description of the file (not a log message -- that happens at
check-in).  New directories must also be added with {\tt cvs add}, but no
description is requested.

The command to check-in changed files is {\tt cvs ci}.  As with {\tt cvs diff},
CVS will accept particular file names or search recursively through the directory
tree looking for files that have been modified.  CVS prompts the user for a
log message for the files being checked in.  If the specific filenames are listed on
the command at
check-in, only a single log message that applies to all of them is required.  
If CVS must search and compare to find the files that are being checked in with
changes, it prompts for a log message for
all of the modified files in a given directory.  The {\tt EDITOR}
environmental variable is used to decide which editor to bring up to
enter the log message.

CVS automatically tracks which version(s) of the source a newly checked-in working
copy is based on. This allows it to determine whether the changes would be
checked in on a branch or the main trunk, etc.

To delete an entire working directory, the simplest approach is to use the command
{\tt cvs release -d nwchem} in the directory above it.  This command
checks the files in the working directory, looking for changes that have not
yet been checked back into the repository.  This is to ensure that changes
are not accidentally abandoned.  If no inconsistencies are found, CVS
deletes the entire
directory tree.   (NOTE: leaving off the {\tt -d} just does the check without
deleting anything.)

The above commands provide a convenient starting point for learning how to use CVS.
However, users wishing to obtain a more thorough understanding of the capabilities
of the system should read through the CVS man pages to get a better feel 
for everything that can
be done.  (Hint: If you are unsure of what a command will do, try it first
with a {\tt -n} option on cvs itself.  This is like ``make -n'', which
reports what it would do if invoked without the {\tt -n}.  But it does not actually
do anything.  Honest.)

\subsection{Summary of CVS commands}

The following is provided as a quick reference guide to CVS.  A more
detailed short-form reference is available in
\texttt{nwchem/doc/cvshelp.man}.  Detailed documentation can be obtained using 
the command
\texttt{man cvs}.

\begin{description}
\item{\texttt{setenv CVSROOT /msrc/proj/mss}} --- in \texttt{csh} this
  defines the path to the CVS repository.  Put this in your
  \texttt{.cshrc} or \texttt{.mycshrc}.
 
\item{\texttt{cvs co nwchem}} --- checks out the entire source for NWChem into
  the directory \texttt{nwchem}.  The repository is unaffected.
  
\item{\texttt{cvs -n update}} --- compares the contents of the current
  directory and all subdirectories against the repository and flags
  files according to their status:
  \begin{description}
  \item{\texttt{?}} --- the file is not maintained by CVS.
  \item{\texttt{M}} --- your checked-out version differs from the original
    (i.e., you edited it).
  \item{\texttt{U}} --- your checked-out version is out-of-date and
    needs updating.
  \item{\texttt{C}} --- potential conflict. You have changed this file
    and the source in the repository has also changed.
  \item{File not listed} --- your source is the same as that in the repository.
  \end{description}
  Neither the repository nor your source are changed.

\item{\texttt{cvs update}} --- updates the contents of the current
  directory and all subdirectories with the latest versions of the
  source, again flagging files according to their status. {\em You are
    responsible for correcting files that CVS flags as containing
    conflicts between edits you and others have made.} However, CVS
  handles all other merging.  New files will also be added to your
  source, but to get new directories you must append the
  \texttt{-d} flag.  Your source is changed; the repository is
  unaffected.

\item{\texttt{cvs diff filename}} --- generates differences between the
  file and the version of the file you checked out (i.e., it indicates
  edits you made).  If you want to compare against the most recent
  version in the repository use \texttt{cvs diff -r head filename}.
  Neither the repository nor your source are changed.

\item{\texttt{cvs add filename}} --- adds a new file to the repository.
  The new file is not actually added until you execute \texttt{cvs
 commit}.  Changes CVS internal information in your source tree but
  does not affect the repository.

\item{\texttt{cvs rm filename}} --- to delete a file from the repository
  delete it from your source with the standard UNIX \texttt{rm} command
  then tell CVS to delete it with this command.  The file
  is not actually removed until you execute \texttt{cvs commit}.  Changes CVS
  internal information in your source tree but does not affect the
  repository.
  
\item{\texttt{cvs commit}} --- this is the only command that affects the
  repository.  Before committing changes and updating the repository
  with changes in a list of files or the current directory tree you
  must
  \begin{itemize}
  \item ensure that all of your sources are up-to-date with respect to
    the repository by using \texttt{cvs update},
  \item resolve all conflicts resulting from the update, and
  \item ensure that the updated code functions correctly.
  \end{itemize}
  Commit will verify that all source is up-to-date before proceeding.
  Then it will prompt (using an editor) for log messages describing
  the changes made.  Be as detailed as possible.
\end{description}

\subsection{Troubleshooting CVS}

{\em Under no circumstances edit, move, delete or otherwise mess with
  files in the NWChem repository.}
  Contact NWChem support at {\tt nwchem-support@emsl.pnl.gov}
  to report problems.

CVS version information is 
"sticky".  That is, CVS usually remembers the specific version checked out
to a working directory.  This can be confusing, 
since the output of such commands as \texttt{cvs update}, etc., will not 
always refer to the latest (or
head) version.  Changes can magically
disappear.  This may be desirable. Or it may not be.  The option \texttt{-A} 
forces the system to look at the lastest version when doing the update.
The form of the command is

\begin{verbatim}
cvs update -A
\end{verbatim}

If CVS is interrupted, or there is an AFS to NFS translator problem,
it may occasionally leave locked files in the CVS repository, causing
subsequent commands to wait forever, printing messages indicating it is
waiting for someone to relinquish access to a specific directory.
Fixing this requires deleting files from the repository.
Contact {\tt nwchem-support@emsl.pnl.gov} for help.

It is unclear if this next problem still exists within EMSL but it may
arise elsewhere.  Because of a problem with the AFS version of the {\tt ci}
command, which is used by CVS, {\tt
  /usr/local/lib/rcs/diff} must be available on the system.  
The easiest way to do this
is to create the {\tt /usr/local/lib/rcs directory} and put in it a
symbolic link to the GNU diff program, {\tt /msrc/bin/diff}.
